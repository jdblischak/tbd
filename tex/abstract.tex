\abstract
A major goal of human genetics is to characterize the role of genetic
variation on complex, polygenic phenotypes. With the discovery from
genome-wide association studies (GWAS) that many associated variants
have a small effect size and are located in non-coding regions of the
genome, there has been a large effort to collect functional genomics
data. The hope is that a better understanding of how the genome
functions in diverse developmental states and environments will provide
insight into the context-specific activity of associated non-coding
variants. My research applies this paradigm to the complex phenotype of
susceptibility to develop tuberculosis (TB). It has been estimated that
10\% of individuals infected with \emph{Mycobacterium tuberculosis}
(MTB) progress to active disease. Despite being heritable, very few
genetic variants have been associated with susceptibility to TB. For my
studies, I have used RNA-sequencing to interrogate genome-wide
transcript levels in \emph{in vitro} cellular models. In addition to
providing insight into the genes important for fighting MTB, I have
pioneered the use of advanced modeling frameworks for analyzing complex,
multivariate data sets and developed pipelines for processing data from
new genomics technologies.
